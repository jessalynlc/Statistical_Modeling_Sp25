% Options for packages loaded elsewhere
\PassOptionsToPackage{unicode}{hyperref}
\PassOptionsToPackage{hyphens}{url}
%
\documentclass[
]{article}
\usepackage{amsmath,amssymb}
\usepackage{iftex}
\ifPDFTeX
  \usepackage[T1]{fontenc}
  \usepackage[utf8]{inputenc}
  \usepackage{textcomp} % provide euro and other symbols
\else % if luatex or xetex
  \usepackage{unicode-math} % this also loads fontspec
  \defaultfontfeatures{Scale=MatchLowercase}
  \defaultfontfeatures[\rmfamily]{Ligatures=TeX,Scale=1}
\fi
\usepackage{lmodern}
\ifPDFTeX\else
  % xetex/luatex font selection
\fi
% Use upquote if available, for straight quotes in verbatim environments
\IfFileExists{upquote.sty}{\usepackage{upquote}}{}
\IfFileExists{microtype.sty}{% use microtype if available
  \usepackage[]{microtype}
  \UseMicrotypeSet[protrusion]{basicmath} % disable protrusion for tt fonts
}{}
\makeatletter
\@ifundefined{KOMAClassName}{% if non-KOMA class
  \IfFileExists{parskip.sty}{%
    \usepackage{parskip}
  }{% else
    \setlength{\parindent}{0pt}
    \setlength{\parskip}{6pt plus 2pt minus 1pt}}
}{% if KOMA class
  \KOMAoptions{parskip=half}}
\makeatother
\usepackage{xcolor}
\usepackage[margin=2.54cm]{geometry}
\usepackage{color}
\usepackage{fancyvrb}
\newcommand{\VerbBar}{|}
\newcommand{\VERB}{\Verb[commandchars=\\\{\}]}
\DefineVerbatimEnvironment{Highlighting}{Verbatim}{commandchars=\\\{\}}
% Add ',fontsize=\small' for more characters per line
\usepackage{framed}
\definecolor{shadecolor}{RGB}{248,248,248}
\newenvironment{Shaded}{\begin{snugshade}}{\end{snugshade}}
\newcommand{\AlertTok}[1]{\textcolor[rgb]{0.94,0.16,0.16}{#1}}
\newcommand{\AnnotationTok}[1]{\textcolor[rgb]{0.56,0.35,0.01}{\textbf{\textit{#1}}}}
\newcommand{\AttributeTok}[1]{\textcolor[rgb]{0.13,0.29,0.53}{#1}}
\newcommand{\BaseNTok}[1]{\textcolor[rgb]{0.00,0.00,0.81}{#1}}
\newcommand{\BuiltInTok}[1]{#1}
\newcommand{\CharTok}[1]{\textcolor[rgb]{0.31,0.60,0.02}{#1}}
\newcommand{\CommentTok}[1]{\textcolor[rgb]{0.56,0.35,0.01}{\textit{#1}}}
\newcommand{\CommentVarTok}[1]{\textcolor[rgb]{0.56,0.35,0.01}{\textbf{\textit{#1}}}}
\newcommand{\ConstantTok}[1]{\textcolor[rgb]{0.56,0.35,0.01}{#1}}
\newcommand{\ControlFlowTok}[1]{\textcolor[rgb]{0.13,0.29,0.53}{\textbf{#1}}}
\newcommand{\DataTypeTok}[1]{\textcolor[rgb]{0.13,0.29,0.53}{#1}}
\newcommand{\DecValTok}[1]{\textcolor[rgb]{0.00,0.00,0.81}{#1}}
\newcommand{\DocumentationTok}[1]{\textcolor[rgb]{0.56,0.35,0.01}{\textbf{\textit{#1}}}}
\newcommand{\ErrorTok}[1]{\textcolor[rgb]{0.64,0.00,0.00}{\textbf{#1}}}
\newcommand{\ExtensionTok}[1]{#1}
\newcommand{\FloatTok}[1]{\textcolor[rgb]{0.00,0.00,0.81}{#1}}
\newcommand{\FunctionTok}[1]{\textcolor[rgb]{0.13,0.29,0.53}{\textbf{#1}}}
\newcommand{\ImportTok}[1]{#1}
\newcommand{\InformationTok}[1]{\textcolor[rgb]{0.56,0.35,0.01}{\textbf{\textit{#1}}}}
\newcommand{\KeywordTok}[1]{\textcolor[rgb]{0.13,0.29,0.53}{\textbf{#1}}}
\newcommand{\NormalTok}[1]{#1}
\newcommand{\OperatorTok}[1]{\textcolor[rgb]{0.81,0.36,0.00}{\textbf{#1}}}
\newcommand{\OtherTok}[1]{\textcolor[rgb]{0.56,0.35,0.01}{#1}}
\newcommand{\PreprocessorTok}[1]{\textcolor[rgb]{0.56,0.35,0.01}{\textit{#1}}}
\newcommand{\RegionMarkerTok}[1]{#1}
\newcommand{\SpecialCharTok}[1]{\textcolor[rgb]{0.81,0.36,0.00}{\textbf{#1}}}
\newcommand{\SpecialStringTok}[1]{\textcolor[rgb]{0.31,0.60,0.02}{#1}}
\newcommand{\StringTok}[1]{\textcolor[rgb]{0.31,0.60,0.02}{#1}}
\newcommand{\VariableTok}[1]{\textcolor[rgb]{0.00,0.00,0.00}{#1}}
\newcommand{\VerbatimStringTok}[1]{\textcolor[rgb]{0.31,0.60,0.02}{#1}}
\newcommand{\WarningTok}[1]{\textcolor[rgb]{0.56,0.35,0.01}{\textbf{\textit{#1}}}}
\usepackage{graphicx}
\makeatletter
\def\maxwidth{\ifdim\Gin@nat@width>\linewidth\linewidth\else\Gin@nat@width\fi}
\def\maxheight{\ifdim\Gin@nat@height>\textheight\textheight\else\Gin@nat@height\fi}
\makeatother
% Scale images if necessary, so that they will not overflow the page
% margins by default, and it is still possible to overwrite the defaults
% using explicit options in \includegraphics[width, height, ...]{}
\setkeys{Gin}{width=\maxwidth,height=\maxheight,keepaspectratio}
% Set default figure placement to htbp
\makeatletter
\def\fps@figure{htbp}
\makeatother
\setlength{\emergencystretch}{3em} % prevent overfull lines
\providecommand{\tightlist}{%
  \setlength{\itemsep}{0pt}\setlength{\parskip}{0pt}}
\setcounter{secnumdepth}{-\maxdimen} % remove section numbering
\usepackage{booktabs}
\usepackage{caption}
\usepackage{longtable}
\usepackage{colortbl}
\usepackage{array}
\usepackage{anyfontsize}
\usepackage{multirow}
\ifLuaTeX
  \usepackage{selnolig}  % disable illegal ligatures
\fi
\usepackage{bookmark}
\IfFileExists{xurl.sty}{\usepackage{xurl}}{} % add URL line breaks if available
\urlstyle{same}
\hypersetup{
  pdftitle={Week 7 Assignment},
  pdfauthor={Jessalyn Chuang},
  hidelinks,
  pdfcreator={LaTeX via pandoc}}

\title{Week 7 Assignment}
\author{Jessalyn Chuang}
\date{}

\begin{document}
\maketitle

\begin{Shaded}
\begin{Highlighting}[]
\CommentTok{\# Load necessary packages.}
\FunctionTok{library}\NormalTok{(here)}
\end{Highlighting}
\end{Shaded}

\begin{verbatim}
## here() starts at /home/guest/Statistical_Modeling_Sp25
\end{verbatim}

\begin{Shaded}
\begin{Highlighting}[]
\FunctionTok{library}\NormalTok{(tidyverse)}
\end{Highlighting}
\end{Shaded}

\begin{verbatim}
## -- Attaching core tidyverse packages ------------------------ tidyverse 2.0.0 --
## v dplyr     1.1.4     v readr     2.1.5
## v forcats   1.0.0     v stringr   1.5.1
## v ggplot2   3.5.1     v tibble    3.2.1
## v lubridate 1.9.3     v tidyr     1.3.1
## v purrr     1.0.2
\end{verbatim}

\begin{verbatim}
## -- Conflicts ------------------------------------------ tidyverse_conflicts() --
## x dplyr::filter() masks stats::filter()
## x dplyr::lag()    masks stats::lag()
## i Use the conflicted package (<http://conflicted.r-lib.org/>) to force all conflicts to become errors
\end{verbatim}

\begin{Shaded}
\begin{Highlighting}[]
\CommentTok{\#install.packages(\textquotesingle{}dunn.test\textquotesingle{})}
\FunctionTok{library}\NormalTok{(dunn.test)}
\FunctionTok{library}\NormalTok{(gt)}
\CommentTok{\#install.packages(\textquotesingle{}paletteer\textquotesingle{})}
\FunctionTok{library}\NormalTok{(paletteer)}
\CommentTok{\#install.packages("tibble")}
\FunctionTok{library}\NormalTok{(tibble)}
\end{Highlighting}
\end{Shaded}

\#(1) Mann-Whitney U and Wilcoxon Signed-Rank Tests

\begin{enumerate}
\def\labelenumi{\alph{enumi}.}
\tightlist
\item
  Filter the dataset only for cutthroat trout. Create a figure that
  helps you to demonstrate why their snout-to-fork length
  (length\_1\_mm) by forest treatment (section) is not normally
  distributed. Clear-cut regions of the forest are denoted with CC, and
  old-growth regions of the forest are denoted as OG. Run a Mann-Whitney
  U test to see if ranks of fish lengths are significantly different
  between forest treatment types. Write out your findings as you would
  in a report, including measures of central tendency and data spread
  for both forest treatments.
\end{enumerate}

\begin{Shaded}
\begin{Highlighting}[]
\NormalTok{vertebrates\_data }\OtherTok{\textless{}{-}} \FunctionTok{read.csv}\NormalTok{(}\AttributeTok{file =} \FunctionTok{here}\NormalTok{(}\StringTok{"./Assignments/week7/and\_vertebrates.csv"}\NormalTok{),}
                           \AttributeTok{stringsAsFactors =} \ConstantTok{TRUE}\NormalTok{)}

\NormalTok{trout\_data }\OtherTok{\textless{}{-}} \FunctionTok{filter}\NormalTok{(vertebrates\_data, }
\NormalTok{                              species }\SpecialCharTok{==} \StringTok{"Cutthroat trout"}\NormalTok{)}

\NormalTok{trout\_fig1 }\OtherTok{\textless{}{-}} \FunctionTok{ggplot}\NormalTok{(trout\_data, }\FunctionTok{aes}\NormalTok{(}\AttributeTok{x =}\NormalTok{ length\_1\_mm, }\AttributeTok{fill =}\NormalTok{ section)) }\SpecialCharTok{+}
  \FunctionTok{geom\_histogram}\NormalTok{() }\SpecialCharTok{+}
  \FunctionTok{scale\_fill\_manual}\NormalTok{(}\AttributeTok{values =} \FunctionTok{c}\NormalTok{(}\StringTok{"darkorange"}\NormalTok{, }\StringTok{"pink"}\NormalTok{, }\StringTok{"purple"}\NormalTok{)) }\SpecialCharTok{+}
  \FunctionTok{labs}\NormalTok{(}\AttributeTok{x =} \StringTok{"Snout{-}to{-}Fork Length (mm)"}\NormalTok{, }\AttributeTok{y =} \StringTok{"Count"}\NormalTok{) }\SpecialCharTok{+}
  \FunctionTok{facet\_grid}\NormalTok{(.}\SpecialCharTok{\textasciitilde{}}\NormalTok{section) }\SpecialCharTok{+}
  \FunctionTok{theme\_bw}\NormalTok{() }\SpecialCharTok{+}
  \FunctionTok{theme}\NormalTok{(}\AttributeTok{legend.position =} \StringTok{"none"}\NormalTok{)}

\NormalTok{trout\_fig1}
\end{Highlighting}
\end{Shaded}

\begin{verbatim}
## `stat_bin()` using `bins = 30`. Pick better value with `binwidth`.
\end{verbatim}

\begin{verbatim}
## Warning: Removed 5 rows containing non-finite outside the scale range
## (`stat_bin()`).
\end{verbatim}

\includegraphics{week7assignment_JC_files/figure-latex/unnamed-chunk-2-1.pdf}

\begin{Shaded}
\begin{Highlighting}[]
\CommentTok{\# Save data by section type.}
\NormalTok{trout\_CC }\OtherTok{\textless{}{-}}\NormalTok{ trout\_data }\SpecialCharTok{\%\textgreater{}\%}
\FunctionTok{filter}\NormalTok{(section }\SpecialCharTok{==} \StringTok{"CC"}\NormalTok{)}
\NormalTok{trout\_OG }\OtherTok{\textless{}{-}}\NormalTok{ trout\_data }\SpecialCharTok{\%\textgreater{}\%}
\FunctionTok{filter}\NormalTok{(section }\SpecialCharTok{==} \StringTok{"OG"}\NormalTok{)}

\NormalTok{mann\_test }\OtherTok{\textless{}{-}} \FunctionTok{wilcox.test}\NormalTok{(trout\_CC}\SpecialCharTok{$}\NormalTok{length\_1\_mm, trout\_OG}\SpecialCharTok{$}\NormalTok{length\_1\_mm,}
\AttributeTok{paired =} \ConstantTok{FALSE}\NormalTok{)}

\NormalTok{mann\_test}
\end{Highlighting}
\end{Shaded}

\begin{verbatim}
## 
##  Wilcoxon rank sum test with continuity correction
## 
## data:  trout_CC$length_1_mm and trout_OG$length_1_mm
## W = 55178843, p-value = 7.697e-16
## alternative hypothesis: true location shift is not equal to 0
\end{verbatim}

\begin{Shaded}
\begin{Highlighting}[]
\NormalTok{trout\_CC\_mean }\OtherTok{\textless{}{-}} \FunctionTok{mean}\NormalTok{(trout\_CC}\SpecialCharTok{$}\NormalTok{length\_1\_mm, }\AttributeTok{na.rm =} \ConstantTok{TRUE}\NormalTok{)}
\NormalTok{trout\_CC\_sd }\OtherTok{\textless{}{-}} \FunctionTok{sd}\NormalTok{(trout\_CC}\SpecialCharTok{$}\NormalTok{length\_1\_mm, }\AttributeTok{na.rm =} \ConstantTok{TRUE}\NormalTok{)}

\NormalTok{trout\_OG\_mean }\OtherTok{\textless{}{-}} \FunctionTok{mean}\NormalTok{(trout\_OG}\SpecialCharTok{$}\NormalTok{length\_1\_mm, }\AttributeTok{na.rm =} \ConstantTok{TRUE}\NormalTok{)}
\NormalTok{trout\_OG\_sd }\OtherTok{\textless{}{-}} \FunctionTok{sd}\NormalTok{(trout\_OG}\SpecialCharTok{$}\NormalTok{length\_1\_mm, }\AttributeTok{na.rm =} \ConstantTok{TRUE}\NormalTok{)}
\end{Highlighting}
\end{Shaded}

Cutthroat trout species displayed significant differences in
Snout-to-Fork length as determined by a Mann-Whitney U-Test (W =
55178843, p \textless{} 0.001). Therefore, we accept the alternative
hypothesis that the ranks of the Snout-to-Fork length for Cutthroat
trout in Clear-Cut regions (mean = 85.31 mm, s.d. = 36.03 mm) and
Old-Growth regions (mean = 981.43 mm, s.d. = 34.46 mm) are significantly
different.

\begin{enumerate}
\def\labelenumi{\alph{enumi}.}
\setcounter{enumi}{1}
\tightlist
\item
  Navigate to Canvas and download the trout\_recapture.csv dataset;
  these data are originally part of the and\_vertebrates.csv dataset,
  but they have been filtered to include only trout that were captured
  and tagged in 2014 and then re-captured in 2015. Create a figure that
  helps you demonstrate why their snout-to-fork length (length\_1\_mm)
  by year is not normally distributed. Run a Wilcoxon Signed-Rank test
  to see if ranks of fish lengths are significantly different between
  years. Write out your findings as you would in a report, including
  measures of central tendency and data spread for both years.
\end{enumerate}

\begin{Shaded}
\begin{Highlighting}[]
\NormalTok{trout\_recapture }\OtherTok{\textless{}{-}} \FunctionTok{read.csv}\NormalTok{(}\AttributeTok{file =} \FunctionTok{here}\NormalTok{(}\StringTok{"./Assignments/week7/trout\_recapture\_v2.csv"}\NormalTok{),}
                           \AttributeTok{stringsAsFactors =} \ConstantTok{TRUE}\NormalTok{)}

\NormalTok{trout\_fig2 }\OtherTok{\textless{}{-}} \FunctionTok{ggplot}\NormalTok{(trout\_recapture, }\FunctionTok{aes}\NormalTok{(}\AttributeTok{x =}\NormalTok{ length\_1\_mm, }\AttributeTok{fill =} \FunctionTok{factor}\NormalTok{(year))) }\SpecialCharTok{+}
  \FunctionTok{geom\_histogram}\NormalTok{() }\SpecialCharTok{+}
  \FunctionTok{scale\_fill\_manual}\NormalTok{(}\AttributeTok{values =} \FunctionTok{c}\NormalTok{(}\StringTok{"darkorange"}\NormalTok{, }\StringTok{"pink"}\NormalTok{)) }\SpecialCharTok{+}
  \FunctionTok{labs}\NormalTok{(}\AttributeTok{x =} \StringTok{"Snout{-}to{-}Fork Length (mm)"}\NormalTok{, }\AttributeTok{y =} \StringTok{"Count"}\NormalTok{) }\SpecialCharTok{+}
  \FunctionTok{facet\_grid}\NormalTok{(.}\SpecialCharTok{\textasciitilde{}}\NormalTok{year) }\SpecialCharTok{+}
  \FunctionTok{theme\_bw}\NormalTok{() }\SpecialCharTok{+}
  \FunctionTok{theme}\NormalTok{(}\AttributeTok{legend.position =} \StringTok{"none"}\NormalTok{)}

\NormalTok{trout\_fig2}
\end{Highlighting}
\end{Shaded}

\begin{verbatim}
## `stat_bin()` using `bins = 30`. Pick better value with `binwidth`.
\end{verbatim}

\includegraphics{week7assignment_JC_files/figure-latex/unnamed-chunk-3-1.pdf}

\begin{Shaded}
\begin{Highlighting}[]
\CommentTok{\# Save data by year.}
\NormalTok{trout\_2014 }\OtherTok{\textless{}{-}}\NormalTok{ trout\_recapture }\SpecialCharTok{\%\textgreater{}\%}
\FunctionTok{filter}\NormalTok{(year }\SpecialCharTok{==} \DecValTok{2014}\NormalTok{)}
\NormalTok{trout\_2015 }\OtherTok{\textless{}{-}}\NormalTok{ trout\_recapture }\SpecialCharTok{\%\textgreater{}\%}
\FunctionTok{filter}\NormalTok{(year }\SpecialCharTok{==} \DecValTok{2015}\NormalTok{)}

\NormalTok{wilcox\_test }\OtherTok{\textless{}{-}} \FunctionTok{wilcox.test}\NormalTok{(trout\_2014}\SpecialCharTok{$}\NormalTok{length\_1\_mm, trout\_2015}\SpecialCharTok{$}\NormalTok{length\_1\_mm,}
\AttributeTok{paired =} \ConstantTok{TRUE}\NormalTok{)}

\NormalTok{wilcox\_test}
\end{Highlighting}
\end{Shaded}

\begin{verbatim}
## 
##  Wilcoxon signed rank test with continuity correction
## 
## data:  trout_2014$length_1_mm and trout_2015$length_1_mm
## V = 625.5, p-value = 4.973e-08
## alternative hypothesis: true location shift is not equal to 0
\end{verbatim}

\begin{Shaded}
\begin{Highlighting}[]
\NormalTok{trout\_2014\_mean }\OtherTok{\textless{}{-}} \FunctionTok{mean}\NormalTok{(trout\_2014}\SpecialCharTok{$}\NormalTok{length\_1\_mm, }\AttributeTok{na.rm =} \ConstantTok{TRUE}\NormalTok{)}
\NormalTok{trout\_2014\_sd }\OtherTok{\textless{}{-}} \FunctionTok{sd}\NormalTok{(trout\_2014}\SpecialCharTok{$}\NormalTok{length\_1\_mm, }\AttributeTok{na.rm =} \ConstantTok{TRUE}\NormalTok{)}

\NormalTok{trout\_2015\_mean }\OtherTok{\textless{}{-}} \FunctionTok{mean}\NormalTok{(trout\_2015}\SpecialCharTok{$}\NormalTok{length\_1\_mm, }\AttributeTok{na.rm =} \ConstantTok{TRUE}\NormalTok{)}
\NormalTok{trout\_2015\_sd }\OtherTok{\textless{}{-}} \FunctionTok{sd}\NormalTok{(trout\_2015}\SpecialCharTok{$}\NormalTok{length\_1\_mm, }\AttributeTok{na.rm =} \ConstantTok{TRUE}\NormalTok{)}
\end{Highlighting}
\end{Shaded}

Recaptured Cutthroat trout displayed significant differences in
Snout-to-Fork length as determined by a Wilcoxon Signed-Rank Test (V =
625.5, p \textless{} 0.001). Therefore, we accept the alternative
hypothesis that the ranks of the Snout-to-Fork length for Cutthroat
trout in 2014 (mean = 105.48 mm, s.d. = 17.27 mm) and 2015 (mean =
119.63 mm, s.d. = 14.84 mm) are significantly different.

\#(2) Kruskal Wallis and post-hoc Dunn's Tests

\begin{enumerate}
\def\labelenumi{\alph{enumi}.}
\tightlist
\item
  Using the dataset you created above including only cutthroat trout,
  create a figure that helps you to demonstrate why their snout-to-fork
  length (length\_1\_mm) by reach (reach) is not normally distributed.
  Having verified this, run a Kruskal Wallis test to see if ranks of
  fish lengths are significantly different among stream reaches. If you
  find there is a significant difference between at least two reaches,
  run a Dunn's test. Write out your findings as you would in a report,
  including a measure of central tendency and data spread for all three
  reach sections.
\end{enumerate}

\begin{Shaded}
\begin{Highlighting}[]
\NormalTok{trout\_fig3 }\OtherTok{\textless{}{-}} \FunctionTok{ggplot}\NormalTok{(trout\_data, }\FunctionTok{aes}\NormalTok{(}\AttributeTok{x =}\NormalTok{ length\_1\_mm, }\AttributeTok{fill =}\NormalTok{ reach)) }\SpecialCharTok{+}
  \FunctionTok{geom\_histogram}\NormalTok{() }\SpecialCharTok{+}
  \FunctionTok{scale\_fill\_manual}\NormalTok{(}\AttributeTok{values =} \FunctionTok{c}\NormalTok{(}\StringTok{"darkorange"}\NormalTok{, }\StringTok{"pink"}\NormalTok{, }\StringTok{"purple"}\NormalTok{)) }\SpecialCharTok{+}
  \FunctionTok{labs}\NormalTok{(}\AttributeTok{x =} \StringTok{"Snout{-}to{-}Fork Length (mm)"}\NormalTok{, }\AttributeTok{y =} \StringTok{"Count"}\NormalTok{) }\SpecialCharTok{+}
  \FunctionTok{facet\_grid}\NormalTok{(.}\SpecialCharTok{\textasciitilde{}}\NormalTok{reach) }\SpecialCharTok{+}
  \FunctionTok{theme\_bw}\NormalTok{() }\SpecialCharTok{+}
  \FunctionTok{theme}\NormalTok{(}\AttributeTok{legend.position =} \StringTok{"none"}\NormalTok{)}

\NormalTok{trout\_fig3}
\end{Highlighting}
\end{Shaded}

\begin{verbatim}
## `stat_bin()` using `bins = 30`. Pick better value with `binwidth`.
\end{verbatim}

\begin{verbatim}
## Warning: Removed 5 rows containing non-finite outside the scale range
## (`stat_bin()`).
\end{verbatim}

\includegraphics{week7assignment_JC_files/figure-latex/unnamed-chunk-4-1.pdf}

\begin{Shaded}
\begin{Highlighting}[]
\NormalTok{kruskal\_test }\OtherTok{\textless{}{-}} \FunctionTok{kruskal.test}\NormalTok{(length\_1\_mm }\SpecialCharTok{\textasciitilde{}}\NormalTok{ reach, }\AttributeTok{data =}\NormalTok{ trout\_data)}

\NormalTok{kruskal\_test}
\end{Highlighting}
\end{Shaded}

\begin{verbatim}
## 
##  Kruskal-Wallis rank sum test
## 
## data:  length_1_mm by reach
## Kruskal-Wallis chi-squared = 77.394, df = 2, p-value < 2.2e-16
\end{verbatim}

\begin{Shaded}
\begin{Highlighting}[]
\CommentTok{\# There is a significant difference between at least two reaches}
\NormalTok{dunn\_test }\OtherTok{\textless{}{-}} \FunctionTok{dunn.test}\NormalTok{(trout\_data}\SpecialCharTok{$}\NormalTok{length\_1\_mm, trout\_data}\SpecialCharTok{$}\NormalTok{reach)}
\end{Highlighting}
\end{Shaded}

\begin{verbatim}
##   Kruskal-Wallis rank sum test
## 
## data: x and group
## Kruskal-Wallis chi-squared = 77.3942, df = 2, p-value = 0
## 
## 
##                            Comparison of x by group                            
##                                 (No adjustment)                                
## Col Mean-|
## Row Mean |          L          M
## ---------+----------------------
##        M |  -8.305134
##          |    0.0000*
##          |
##        U |  -1.688937   6.770059
##          |     0.0456    0.0000*
## 
## alpha = 0.05
## Reject Ho if p <= alpha/2
\end{verbatim}

\begin{Shaded}
\begin{Highlighting}[]
\CommentTok{\# Save data by reach.}
\NormalTok{trout\_L }\OtherTok{\textless{}{-}}\NormalTok{ trout\_data }\SpecialCharTok{\%\textgreater{}\%}
\FunctionTok{filter}\NormalTok{(reach }\SpecialCharTok{==} \StringTok{"L"}\NormalTok{)}
\NormalTok{trout\_M }\OtherTok{\textless{}{-}}\NormalTok{ trout\_data }\SpecialCharTok{\%\textgreater{}\%}
\FunctionTok{filter}\NormalTok{(reach }\SpecialCharTok{==} \StringTok{"M"}\NormalTok{)}
\NormalTok{trout\_U }\OtherTok{\textless{}{-}}\NormalTok{ trout\_data }\SpecialCharTok{\%\textgreater{}\%}
\FunctionTok{filter}\NormalTok{(reach }\SpecialCharTok{==} \StringTok{"U"}\NormalTok{)}

\NormalTok{trout\_L\_mean }\OtherTok{\textless{}{-}} \FunctionTok{mean}\NormalTok{(trout\_L}\SpecialCharTok{$}\NormalTok{length\_1\_mm, }\AttributeTok{na.rm =} \ConstantTok{TRUE}\NormalTok{)}
\NormalTok{trout\_L\_sd }\OtherTok{\textless{}{-}} \FunctionTok{sd}\NormalTok{(trout\_L}\SpecialCharTok{$}\NormalTok{length\_1\_mm, }\AttributeTok{na.rm =} \ConstantTok{TRUE}\NormalTok{)}

\NormalTok{trout\_M\_mean }\OtherTok{\textless{}{-}} \FunctionTok{mean}\NormalTok{(trout\_M}\SpecialCharTok{$}\NormalTok{length\_1\_mm, }\AttributeTok{na.rm =} \ConstantTok{TRUE}\NormalTok{)}
\NormalTok{trout\_M\_sd }\OtherTok{\textless{}{-}} \FunctionTok{sd}\NormalTok{(trout\_M}\SpecialCharTok{$}\NormalTok{length\_1\_mm, }\AttributeTok{na.rm =} \ConstantTok{TRUE}\NormalTok{)}

\NormalTok{trout\_U\_mean }\OtherTok{\textless{}{-}} \FunctionTok{mean}\NormalTok{(trout\_U}\SpecialCharTok{$}\NormalTok{length\_1\_mm, }\AttributeTok{na.rm =} \ConstantTok{TRUE}\NormalTok{)}
\NormalTok{trout\_U\_sd }\OtherTok{\textless{}{-}} \FunctionTok{sd}\NormalTok{(trout\_U}\SpecialCharTok{$}\NormalTok{length\_1\_mm, }\AttributeTok{na.rm =} \ConstantTok{TRUE}\NormalTok{)}
\end{Highlighting}
\end{Shaded}

Cutthroat trout species displayed significant differences in
Snout-to-Fork length as determined by a Kruskall-Wallis Rank Sum Test
(X²(2) = 77.394, p \textless{} 0.001). Post-hoc testing by Dunn's Test
revealed that ranks of fish lengths are significantly different between
L reach (mean = 81.60 mm, s.d. = 34.71 mm) and M reach (mean = 87.01 mm,
s.d. = 36.09 mm), and between M reach and U reach (mean = 82.19 mm, s.d.
= 35.11 mm). There is no significant difference between L and U reaches.

\begin{enumerate}
\def\labelenumi{\alph{enumi}.}
\setcounter{enumi}{1}
\tightlist
\item
  Create a figure with an appropriate caption to display fish lengths by
  reach, including notation to indicate the results of the Kruskal
  Wallis and Dunn's tests (letters, etc.).
\end{enumerate}

\begin{Shaded}
\begin{Highlighting}[]
\NormalTok{letters\_df }\OtherTok{\textless{}{-}} \FunctionTok{tibble}\NormalTok{(}
  \AttributeTok{reach =} \FunctionTok{c}\NormalTok{(}\StringTok{"L"}\NormalTok{, }\StringTok{"M"}\NormalTok{, }\StringTok{"U"}\NormalTok{),}
  \AttributeTok{Letter =} \FunctionTok{c}\NormalTok{(}\StringTok{"a"}\NormalTok{, }\StringTok{"b"}\NormalTok{, }\StringTok{"a"}\NormalTok{) }\CommentTok{\# Corrected assignment}
\NormalTok{)}

\NormalTok{trout\_summary }\OtherTok{\textless{}{-}}\NormalTok{ trout\_data }\SpecialCharTok{\%\textgreater{}\%}
  \FunctionTok{group\_by}\NormalTok{(reach) }\SpecialCharTok{\%\textgreater{}\%}
  \FunctionTok{summarise}\NormalTok{(}\AttributeTok{mean\_length =} \FunctionTok{mean}\NormalTok{(length\_1\_mm, }\AttributeTok{na.rm =} \ConstantTok{TRUE}\NormalTok{),}
            \AttributeTok{sd\_length =} \FunctionTok{sd}\NormalTok{(length\_1\_mm, }\AttributeTok{na.rm =} \ConstantTok{TRUE}\NormalTok{),}
            \AttributeTok{max\_length =} \FunctionTok{max}\NormalTok{(length\_1\_mm, }\AttributeTok{na.rm =} \ConstantTok{TRUE}\NormalTok{)) }\SpecialCharTok{\%\textgreater{}\%}
  \FunctionTok{left\_join}\NormalTok{(letters\_df, }\AttributeTok{by =} \StringTok{"reach"}\NormalTok{)}

\CommentTok{\# Create a boxplot with significance letters}
\NormalTok{trout\_fig4 }\OtherTok{\textless{}{-}} \FunctionTok{ggplot}\NormalTok{(trout\_data, }\FunctionTok{aes}\NormalTok{(}\AttributeTok{x =}\NormalTok{ reach, }\AttributeTok{y =}\NormalTok{ length\_1\_mm, }\AttributeTok{fill =}\NormalTok{ reach)) }\SpecialCharTok{+}
  \FunctionTok{geom\_boxplot}\NormalTok{() }\SpecialCharTok{+}
  \FunctionTok{geom\_text}\NormalTok{(}\AttributeTok{data =}\NormalTok{ trout\_summary, }\FunctionTok{aes}\NormalTok{(}\AttributeTok{x =}\NormalTok{ reach, }\AttributeTok{y =}\NormalTok{ max\_length }\SpecialCharTok{+} \DecValTok{10}\NormalTok{, }\AttributeTok{label =}\NormalTok{ Letter), }
            \AttributeTok{size =} \DecValTok{6}\NormalTok{, }\AttributeTok{vjust =} \DecValTok{0}\NormalTok{) }\SpecialCharTok{+}  \CommentTok{\# Position letters correctly}
  \FunctionTok{scale\_fill\_manual}\NormalTok{(}\AttributeTok{values =} \FunctionTok{c}\NormalTok{(}\StringTok{"darkorange"}\NormalTok{, }\StringTok{"pink"}\NormalTok{, }\StringTok{"purple"}\NormalTok{)) }\SpecialCharTok{+}
  \FunctionTok{labs}\NormalTok{(}\AttributeTok{x =} \StringTok{"Reach"}\NormalTok{, }\AttributeTok{y =} \StringTok{"Snout{-}to{-}Fork Length (mm)"}\NormalTok{, }\AttributeTok{fill =} \StringTok{"Reach"}\NormalTok{,}
       \AttributeTok{title =} \StringTok{"Fish Length by Reach"}\NormalTok{) }\SpecialCharTok{+}
  \FunctionTok{theme\_bw}\NormalTok{() }\SpecialCharTok{+}
  \FunctionTok{theme}\NormalTok{(}\AttributeTok{legend.position =} \StringTok{"none"}\NormalTok{)}

\CommentTok{\# Display the final plot}
\FunctionTok{print}\NormalTok{(trout\_fig4)}
\end{Highlighting}
\end{Shaded}

\begin{verbatim}
## Warning: Removed 5 rows containing non-finite outside the scale range
## (`stat_boxplot()`).
\end{verbatim}

\includegraphics{week7assignment_JC_files/figure-latex/unnamed-chunk-5-1.pdf}

\begin{Shaded}
\begin{Highlighting}[]
\FunctionTok{cat}\NormalTok{(}\StringTok{"This boxplot shows the distribution of fish length across different reaches (L, M, U). The letters above each box indicate statistical groupings based on Dunn’s post{-}hoc test. Groups sharing the same letter are not significantly different, whereas groups with different letters indicate a significant difference in fish length (alpha = 0.05)."}\NormalTok{)}
\end{Highlighting}
\end{Shaded}

\begin{verbatim}
## This boxplot shows the distribution of fish length across different reaches (L, M, U). The letters above each box indicate statistical groupings based on Dunn’s post-hoc test. Groups sharing the same letter are not significantly different, whereas groups with different letters indicate a significant difference in fish length (alpha = 0.05).
\end{verbatim}

\#(3) Contingency Tables and the Chi-Squared Test

\begin{enumerate}
\def\labelenumi{\alph{enumi}.}
\tightlist
\item
  Using the dataset you created above including only cutthroat trout,
  create a contingency table that includes counts of trout found in
  different channel habitat types (unittype) by reach (reach). If you're
  unsure how to create this kind of table, think back to the group\_by()
  and summarize() functions you learned last week. Also, you may want to
  explore the pivot\_wider() and pivot\_longer() functions for pivoting
  data from long to wide format. Create a table using the gt() function
  with an appropriate caption to display raw fish counts.
\end{enumerate}

\begin{Shaded}
\begin{Highlighting}[]
\NormalTok{trout\_summary\_unittypebyreach }\OtherTok{\textless{}{-}}\NormalTok{ trout\_data }\SpecialCharTok{\%\textgreater{}\%}
  \FunctionTok{group\_by}\NormalTok{(unittype, reach) }\SpecialCharTok{\%\textgreater{}\%}
  \FunctionTok{summarize}\NormalTok{(}\AttributeTok{count =} \FunctionTok{n}\NormalTok{(), }\AttributeTok{.groups =} \StringTok{"drop"}\NormalTok{)}

\NormalTok{habitatbyreach\_table }\OtherTok{\textless{}{-}}\NormalTok{ trout\_summary\_unittypebyreach }\SpecialCharTok{\%\textgreater{}\%}
  \FunctionTok{select}\NormalTok{(unittype, reach, count) }\SpecialCharTok{\%\textgreater{}\%} 
  \FunctionTok{pivot\_wider}\NormalTok{(}\AttributeTok{names\_from =}\NormalTok{ reach, }\AttributeTok{values\_from =}\NormalTok{ count)}

\CommentTok{\# Make this table a dataframe}
\NormalTok{habitatbyreach\_df }\OtherTok{\textless{}{-}} \FunctionTok{as.data.frame}\NormalTok{(habitatbyreach\_table)}

\CommentTok{\# Add row totals}
\NormalTok{habitatbyreach\_df }\OtherTok{\textless{}{-}}\NormalTok{ habitatbyreach\_df }\SpecialCharTok{\%\textgreater{}\%}
  \FunctionTok{mutate}\NormalTok{(}\AttributeTok{Total =} \FunctionTok{rowSums}\NormalTok{(}\FunctionTok{select}\NormalTok{(., L, M, U), }\AttributeTok{na.rm =} \ConstantTok{TRUE}\NormalTok{))}

\NormalTok{habitat\_gt }\OtherTok{\textless{}{-}}\NormalTok{ habitatbyreach\_df }\SpecialCharTok{\%\textgreater{}\%}
  \FunctionTok{gt}\NormalTok{() }\SpecialCharTok{\%\textgreater{}\%}
  \FunctionTok{tab\_header}\NormalTok{(}\AttributeTok{title =} \StringTok{"Fish Habitat Counts by Reach"}\NormalTok{) }\SpecialCharTok{\%\textgreater{}\%} 
  \FunctionTok{cols\_label}\NormalTok{(}
    \AttributeTok{unittype =} \StringTok{"Habitat Type"}\NormalTok{,}
    \AttributeTok{L =} \StringTok{"L Reach"}\NormalTok{,}
    \AttributeTok{M =} \StringTok{"M Reach"}\NormalTok{,}
    \AttributeTok{U =} \StringTok{"U Reach"}\NormalTok{,}
    \AttributeTok{Total =} \StringTok{"Total"} 
\NormalTok{  ) }\SpecialCharTok{\%\textgreater{}\%}
  \FunctionTok{grand\_summary\_rows}\NormalTok{(}
    \AttributeTok{columns =} \FunctionTok{c}\NormalTok{(L, M, U, Total),  }
    \AttributeTok{fns =} \FunctionTok{list}\NormalTok{(}\AttributeTok{Total =} \SpecialCharTok{\textasciitilde{}}\FunctionTok{sum}\NormalTok{(., }\AttributeTok{na.rm =} \ConstantTok{TRUE}\NormalTok{)) }
\NormalTok{  ) }\SpecialCharTok{\%\textgreater{}\%}
  \FunctionTok{data\_color}\NormalTok{(}
    \AttributeTok{columns =} \FunctionTok{c}\NormalTok{(L, M, U),}
    \AttributeTok{palette =} \FunctionTok{c}\NormalTok{(}\StringTok{"\#C0D8F0"}\NormalTok{, }\StringTok{"\#5A7ECB"}\NormalTok{),}
    \AttributeTok{alpha =} \FloatTok{0.75}
\NormalTok{  )}

\CommentTok{\# Display the table}
\NormalTok{habitat\_gt}
\end{Highlighting}
\end{Shaded}

\begingroup
\fontsize{12.0pt}{14.4pt}\selectfont
\begin{longtable}{l|crrrr}
\caption*{
{\large Fish Habitat Counts by Reach}
} \\ 
\toprule
 & Habitat Type & L Reach & M Reach & U Reach & Total \\ 
\midrule\addlinespace[2.5pt]
 & C & {\cellcolor[HTML]{rgba(90,126,203,0.75)}{\textcolor[HTML]{FFFFFF}{3951}}} & {\cellcolor[HTML]{rgba(90,126,203,0.75)}{\textcolor[HTML]{FFFFFF}{2984}}} & {\cellcolor[HTML]{rgba(90,126,203,0.75)}{\textcolor[HTML]{FFFFFF}{4484}}} & 11419 \\ 
 & I & {\cellcolor[HTML]{rgba(192,216,240,0.75)}{\textcolor[HTML]{000000}{7}}} & {\cellcolor[HTML]{rgba(128,128,128,0.75)}{\textcolor[HTML]{FFFFFF}{NA}}} & {\cellcolor[HTML]{rgba(192,216,240,0.75)}{\textcolor[HTML]{000000}{16}}} & 23 \\ 
 & IP & {\cellcolor[HTML]{rgba(190,214,239,0.75)}{\textcolor[HTML]{000000}{79}}} & {\cellcolor[HTML]{rgba(192,216,240,0.75)}{\textcolor[HTML]{000000}{6}}} & {\cellcolor[HTML]{rgba(192,216,240,0.75)}{\textcolor[HTML]{000000}{20}}} & 105 \\ 
 & P & {\cellcolor[HTML]{rgba(163,187,229,0.75)}{\textcolor[HTML]{000000}{1218}}} & {\cellcolor[HTML]{rgba(122,151,214,0.75)}{\textcolor[HTML]{FFFFFF}{2121}}} & {\cellcolor[HTML]{rgba(146,172,223,0.75)}{\textcolor[HTML]{000000}{2131}}} & 5470 \\ 
 & R & {\cellcolor[HTML]{rgba(191,215,240,0.75)}{\textcolor[HTML]{000000}{33}}} & {\cellcolor[HTML]{rgba(189,213,239,0.75)}{\textcolor[HTML]{000000}{94}}} & {\cellcolor[HTML]{rgba(186,210,238,0.75)}{\textcolor[HTML]{000000}{293}}} & 420 \\ 
 & S & {\cellcolor[HTML]{rgba(192,216,240,0.75)}{\textcolor[HTML]{000000}{2}}} & {\cellcolor[HTML]{rgba(128,128,128,0.75)}{\textcolor[HTML]{FFFFFF}{NA}}} & {\cellcolor[HTML]{rgba(192,216,240,0.75)}{\textcolor[HTML]{000000}{7}}} & 9 \\ 
 & SC & {\cellcolor[HTML]{rgba(163,188,229,0.75)}{\textcolor[HTML]{000000}{1205}}} & {\cellcolor[HTML]{rgba(157,182,227,0.75)}{\textcolor[HTML]{000000}{1106}}} & {\cellcolor[HTML]{rgba(191,215,240,0.75)}{\textcolor[HTML]{000000}{66}}} & 2377 \\ 
 & NA & {\cellcolor[HTML]{rgba(187,211,238,0.75)}{\textcolor[HTML]{000000}{220}}} & {\cellcolor[HTML]{rgba(186,210,238,0.75)}{\textcolor[HTML]{000000}{210}}} & {\cellcolor[HTML]{rgba(188,212,239,0.75)}{\textcolor[HTML]{000000}{180}}} & 610 \\ 
\midrule 
\midrule 
Total & — & 6715 & 6521 & 7197 & 20433 \\ 
\bottomrule
\end{longtable}
\endgroup

\begin{enumerate}
\def\labelenumi{\alph{enumi}.}
\setcounter{enumi}{1}
\tightlist
\item
  Before proceeding further, write out your null and alternative
  hypotheses for performing a chi-squared test on this dataset. Then,
  run a chi-squared test on the counts dataset you have created. You may
  want to use the column\_to\_rownames() function to make sure the only
  values in cells of your contingency table are trout counts before
  running your chi-squared test. Write out your findings as you would in
  a report, including language about the habitat type where most fish
  are counted in each reach.
\end{enumerate}

Null hypothesis: There is no significant association between habitat
type and reach for Cutthroat trout.

Alternative hypothesis: There is a significant association between
habitat type and reach for Cutthroat trout.

\begin{Shaded}
\begin{Highlighting}[]
\NormalTok{habitat\_matrix }\OtherTok{\textless{}{-}}\NormalTok{ habitatbyreach\_df }\SpecialCharTok{\%\textgreater{}\%}
  \FunctionTok{filter}\NormalTok{(}\SpecialCharTok{!}\FunctionTok{is.na}\NormalTok{(unittype)) }\SpecialCharTok{\%\textgreater{}\%}  \CommentTok{\# Remove NA rows in unittype}
  \FunctionTok{column\_to\_rownames}\NormalTok{(}\AttributeTok{var =} \StringTok{"unittype"}\NormalTok{) }\SpecialCharTok{\%\textgreater{}\%}  \CommentTok{\# Convert unittype to row names}
  \FunctionTok{select}\NormalTok{(L, M, U) }\SpecialCharTok{\%\textgreater{}\%}  \CommentTok{\# Keep only numeric columns}
  \FunctionTok{replace}\NormalTok{(}\FunctionTok{is.na}\NormalTok{(.), }\DecValTok{0}\NormalTok{)  }\CommentTok{\# Replace NA counts with 0}

\CommentTok{\# Run Chi{-}Square test}
\NormalTok{chi\_test }\OtherTok{\textless{}{-}} \FunctionTok{chisq.test}\NormalTok{(habitat\_matrix)}
\end{Highlighting}
\end{Shaded}

\begin{verbatim}
## Warning in chisq.test(habitat_matrix): Chi-squared approximation may be
## incorrect
\end{verbatim}

\begin{Shaded}
\begin{Highlighting}[]
\NormalTok{chi\_test}
\end{Highlighting}
\end{Shaded}

\begin{verbatim}
## 
##  Pearson's Chi-squared test
## 
## data:  habitat_matrix
## X-squared = 1914.4, df = 12, p-value < 2.2e-16
\end{verbatim}

There is a significant association between habitat and reach for
Cutthroat trout (X²(12) = 1914.4, p\textless0.001).

\end{document}
